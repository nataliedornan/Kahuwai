\documentclass[]{article}
\usepackage{lmodern}
\usepackage{amssymb,amsmath}
\usepackage{ifxetex,ifluatex}
\usepackage{fixltx2e} % provides \textsubscript
\ifnum 0\ifxetex 1\fi\ifluatex 1\fi=0 % if pdftex
  \usepackage[T1]{fontenc}
  \usepackage[utf8]{inputenc}
\else % if luatex or xelatex
  \ifxetex
    \usepackage{mathspec}
  \else
    \usepackage{fontspec}
  \fi
  \defaultfontfeatures{Ligatures=TeX,Scale=MatchLowercase}
\fi
% use upquote if available, for straight quotes in verbatim environments
\IfFileExists{upquote.sty}{\usepackage{upquote}}{}
% use microtype if available
\IfFileExists{microtype.sty}{%
\usepackage{microtype}
\UseMicrotypeSet[protrusion]{basicmath} % disable protrusion for tt fonts
}{}
\usepackage[margin=1in]{geometry}
\usepackage{hyperref}
\hypersetup{unicode=true,
            pdftitle={Wailupe Stormwater Network},
            pdfauthor={Erica Johnson},
            pdfborder={0 0 0},
            breaklinks=true}
\urlstyle{same}  % don't use monospace font for urls
\usepackage{color}
\usepackage{fancyvrb}
\newcommand{\VerbBar}{|}
\newcommand{\VERB}{\Verb[commandchars=\\\{\}]}
\DefineVerbatimEnvironment{Highlighting}{Verbatim}{commandchars=\\\{\}}
% Add ',fontsize=\small' for more characters per line
\usepackage{framed}
\definecolor{shadecolor}{RGB}{248,248,248}
\newenvironment{Shaded}{\begin{snugshade}}{\end{snugshade}}
\newcommand{\AlertTok}[1]{\textcolor[rgb]{0.94,0.16,0.16}{#1}}
\newcommand{\AnnotationTok}[1]{\textcolor[rgb]{0.56,0.35,0.01}{\textbf{\textit{#1}}}}
\newcommand{\AttributeTok}[1]{\textcolor[rgb]{0.77,0.63,0.00}{#1}}
\newcommand{\BaseNTok}[1]{\textcolor[rgb]{0.00,0.00,0.81}{#1}}
\newcommand{\BuiltInTok}[1]{#1}
\newcommand{\CharTok}[1]{\textcolor[rgb]{0.31,0.60,0.02}{#1}}
\newcommand{\CommentTok}[1]{\textcolor[rgb]{0.56,0.35,0.01}{\textit{#1}}}
\newcommand{\CommentVarTok}[1]{\textcolor[rgb]{0.56,0.35,0.01}{\textbf{\textit{#1}}}}
\newcommand{\ConstantTok}[1]{\textcolor[rgb]{0.00,0.00,0.00}{#1}}
\newcommand{\ControlFlowTok}[1]{\textcolor[rgb]{0.13,0.29,0.53}{\textbf{#1}}}
\newcommand{\DataTypeTok}[1]{\textcolor[rgb]{0.13,0.29,0.53}{#1}}
\newcommand{\DecValTok}[1]{\textcolor[rgb]{0.00,0.00,0.81}{#1}}
\newcommand{\DocumentationTok}[1]{\textcolor[rgb]{0.56,0.35,0.01}{\textbf{\textit{#1}}}}
\newcommand{\ErrorTok}[1]{\textcolor[rgb]{0.64,0.00,0.00}{\textbf{#1}}}
\newcommand{\ExtensionTok}[1]{#1}
\newcommand{\FloatTok}[1]{\textcolor[rgb]{0.00,0.00,0.81}{#1}}
\newcommand{\FunctionTok}[1]{\textcolor[rgb]{0.00,0.00,0.00}{#1}}
\newcommand{\ImportTok}[1]{#1}
\newcommand{\InformationTok}[1]{\textcolor[rgb]{0.56,0.35,0.01}{\textbf{\textit{#1}}}}
\newcommand{\KeywordTok}[1]{\textcolor[rgb]{0.13,0.29,0.53}{\textbf{#1}}}
\newcommand{\NormalTok}[1]{#1}
\newcommand{\OperatorTok}[1]{\textcolor[rgb]{0.81,0.36,0.00}{\textbf{#1}}}
\newcommand{\OtherTok}[1]{\textcolor[rgb]{0.56,0.35,0.01}{#1}}
\newcommand{\PreprocessorTok}[1]{\textcolor[rgb]{0.56,0.35,0.01}{\textit{#1}}}
\newcommand{\RegionMarkerTok}[1]{#1}
\newcommand{\SpecialCharTok}[1]{\textcolor[rgb]{0.00,0.00,0.00}{#1}}
\newcommand{\SpecialStringTok}[1]{\textcolor[rgb]{0.31,0.60,0.02}{#1}}
\newcommand{\StringTok}[1]{\textcolor[rgb]{0.31,0.60,0.02}{#1}}
\newcommand{\VariableTok}[1]{\textcolor[rgb]{0.00,0.00,0.00}{#1}}
\newcommand{\VerbatimStringTok}[1]{\textcolor[rgb]{0.31,0.60,0.02}{#1}}
\newcommand{\WarningTok}[1]{\textcolor[rgb]{0.56,0.35,0.01}{\textbf{\textit{#1}}}}
\usepackage{graphicx,grffile}
\makeatletter
\def\maxwidth{\ifdim\Gin@nat@width>\linewidth\linewidth\else\Gin@nat@width\fi}
\def\maxheight{\ifdim\Gin@nat@height>\textheight\textheight\else\Gin@nat@height\fi}
\makeatother
% Scale images if necessary, so that they will not overflow the page
% margins by default, and it is still possible to overwrite the defaults
% using explicit options in \includegraphics[width, height, ...]{}
\setkeys{Gin}{width=\maxwidth,height=\maxheight,keepaspectratio}
\IfFileExists{parskip.sty}{%
\usepackage{parskip}
}{% else
\setlength{\parindent}{0pt}
\setlength{\parskip}{6pt plus 2pt minus 1pt}
}
\setlength{\emergencystretch}{3em}  % prevent overfull lines
\providecommand{\tightlist}{%
  \setlength{\itemsep}{0pt}\setlength{\parskip}{0pt}}
\setcounter{secnumdepth}{0}
% Redefines (sub)paragraphs to behave more like sections
\ifx\paragraph\undefined\else
\let\oldparagraph\paragraph
\renewcommand{\paragraph}[1]{\oldparagraph{#1}\mbox{}}
\fi
\ifx\subparagraph\undefined\else
\let\oldsubparagraph\subparagraph
\renewcommand{\subparagraph}[1]{\oldsubparagraph{#1}\mbox{}}
\fi

%%% Use protect on footnotes to avoid problems with footnotes in titles
\let\rmarkdownfootnote\footnote%
\def\footnote{\protect\rmarkdownfootnote}

%%% Change title format to be more compact
\usepackage{titling}

% Create subtitle command for use in maketitle
\newcommand{\subtitle}[1]{
  \posttitle{
    \begin{center}\large#1\end{center}
    }
}

\setlength{\droptitle}{-2em}

  \title{Wailupe Stormwater Network}
    \pretitle{\vspace{\droptitle}\centering\huge}
  \posttitle{\par}
  \subtitle{This is code to process the stormwater network for use in SWMM.}
  \author{Erica Johnson}
    \preauthor{\centering\large\emph}
  \postauthor{\par}
      \predate{\centering\large\emph}
  \postdate{\par}
    \date{May 7, 2020}


\begin{document}
\maketitle

\hypertarget{code-setup}{%
\subsection{Code Setup}\label{code-setup}}

Read libraries and data file here

\begin{Shaded}
\begin{Highlighting}[]
\CommentTok{#Libraries}
\KeywordTok{library}\NormalTok{(tidyverse)}
\KeywordTok{library}\NormalTok{(dplyr)}
\KeywordTok{library}\NormalTok{(data.table)}
\KeywordTok{library}\NormalTok{(geosphere)}
\KeywordTok{library}\NormalTok{(janitor)}



\CommentTok{#Data with clean names}
\NormalTok{data <-}\StringTok{ }\KeywordTok{read_csv}\NormalTok{(}\StringTok{"sw_network_endpoints_wailupe.csv"}\NormalTok{) }\OperatorTok\StringTok{ }\KeywordTok{clean_names}\NormalTok{()}
\NormalTok{vertices_dt <-}\StringTok{ }\KeywordTok{read_csv}\NormalTok{(}\StringTok{"sw_network_allpoints_wailupe.csv"}\NormalTok{) }\OperatorTok\StringTok{ }\KeywordTok{clean_names}\NormalTok{()}
\end{Highlighting}
\end{Shaded}

\hypertarget{data-tidying}{%
\subsection{Data Tidying}\label{data-tidying}}

Select columns with the data we need to use. Rename them for easy
reference.

\begin{Shaded}
\begin{Highlighting}[]
\NormalTok{network <-}\StringTok{ }\NormalTok{data }\OperatorTok\StringTok{ }
\StringTok{  }\KeywordTok{select}\NormalTok{(}
\NormalTok{  objectid,}
\NormalTok{  point_x,}
\NormalTok{  point_y,}
\NormalTok{  elevation,}
\NormalTok{  subcatch_r,}
\NormalTok{  roughness,}
\NormalTok{  type,}
\NormalTok{  diameter,}
\NormalTok{  width,}
\NormalTok{  height,}
\NormalTok{  type_}\DecValTok{1}
\NormalTok{  ) }\OperatorTok\StringTok{ }
\StringTok{  }\KeywordTok{rename}\NormalTok{(}
    \DataTypeTok{name =}\NormalTok{ objectid,}
    \DataTypeTok{x =}\NormalTok{ point_x,}
    \DataTypeTok{y =}\NormalTok{ point_y,}
    \DataTypeTok{subc =}\NormalTok{ subcatch_r,}
    \DataTypeTok{elevation =}\NormalTok{ elevation,}
    \DataTypeTok{shape =}\NormalTok{ type,}
    \DataTypeTok{structure =}\NormalTok{ type_}\DecValTok{1}
\NormalTok{    ) }\OperatorTok\StringTok{ }
\StringTok{  }\KeywordTok{mutate}\NormalTok{ (}\DataTypeTok{shape =} \KeywordTok{str_replace_all}\NormalTok{(shape, }\StringTok{"Reinforced Concrete Pipe"}\NormalTok{, }\StringTok{"CIRCULAR"}\NormalTok{)) }\OperatorTok\StringTok{ }
\StringTok{  }\KeywordTok{mutate}\NormalTok{ (}\DataTypeTok{shape =} \KeywordTok{str_replace_all}\NormalTok{(shape, }\StringTok{"Box Culvert"}\NormalTok{, }\StringTok{"RECT_CLOSED"}\NormalTok{)) }\OperatorTok\StringTok{ }
\StringTok{  }\KeywordTok{mutate}\NormalTok{ (}\DataTypeTok{shape =} \KeywordTok{str_replace_all}\NormalTok{(shape, }\StringTok{"Channel"}\NormalTok{, }\StringTok{"TRAPEZOIDAL"}\NormalTok{)) }\OperatorTok\StringTok{ }
\StringTok{  }\KeywordTok{mutate}\NormalTok{ (}\DataTypeTok{shape =} \KeywordTok{str_replace_all}\NormalTok{(shape, }\StringTok{"Ditch"}\NormalTok{, }\StringTok{"RECT_OPEN"}\NormalTok{)) }\OperatorTok\StringTok{ }
\StringTok{  }\KeywordTok{mutate}\NormalTok{ (}\DataTypeTok{shape =} \KeywordTok{str_replace_all}\NormalTok{(shape, }\StringTok{"Other"}\NormalTok{, }\StringTok{"RECT_OPEN"}\NormalTok{)) }\OperatorTok\StringTok{ }
\StringTok{  }\KeywordTok{distinct}\NormalTok{()}
\end{Highlighting}
\end{Shaded}

``Length'' provided by USGS is distance between xy points. This
``length''" is not the actual length of the conduit because it does not
take into consideration height (xyz), so we will use the difference
between lower distance between xy points calcualte the actual length
further down in the code.

We will also re-calculate distance between xy points because different
sources return different values for some of the conduits and some
conduits need to have this distance calculated anyway because it is
blank.

\hypertarget{create-unique-names-for-nodes-with-the-same-x-and-y-coordinates-and-unique-names-for-conduits}{%
\subsection{1. Create unique names for nodes with the same x and y
coordinates, and unique names for
conduits}\label{create-unique-names-for-nodes-with-the-same-x-and-y-coordinates-and-unique-names-for-conduits}}

\begin{Shaded}
\begin{Highlighting}[]
\CommentTok{#Index xy coordinates with unique IDs if different, same IDs if repeated}
\NormalTok{unique <-}\StringTok{ }\NormalTok{network }\OperatorTok\StringTok{ }
\StringTok{  }\KeywordTok{mutate}\NormalTok{(}
    \DataTypeTok{node =} \KeywordTok{group_indices}\NormalTok{(}
\NormalTok{      network, x, y}
\NormalTok{      )}
\NormalTok{    ) }

\CommentTok{#Assign nodes the letter j for "junction" (SWMM terminology) and conduits the letter c for }
\CommentTok{#"conduit" (SWMM terminology)}

\NormalTok{unique}\OperatorTok{$}\NormalTok{c <-}\StringTok{ "C"}
\NormalTok{unique}\OperatorTok{$}\NormalTok{j <-}\StringTok{ "J"}

\CommentTok{#conduits}
\NormalTok{unique}\OperatorTok{$}\NormalTok{name=}\StringTok{ }\KeywordTok{paste}\NormalTok{(unique}\OperatorTok{$}\NormalTok{c,unique}\OperatorTok{$}\NormalTok{name)}

\CommentTok{#remove space}
\NormalTok{unique}\OperatorTok{$}\NormalTok{name <-}\StringTok{ }\KeywordTok{gsub}\NormalTok{(}
  \StringTok{'}\CharTok{\textbackslash{}\textbackslash{}}\StringTok{s+'}\NormalTok{, }\StringTok{''}\NormalTok{, unique}\OperatorTok{$}\NormalTok{name}
\NormalTok{  )}
\CommentTok{#node}
\NormalTok{unique}\OperatorTok{$}\NormalTok{node=}\StringTok{ }\KeywordTok{paste}\NormalTok{(unique}\OperatorTok{$}\NormalTok{j,unique}\OperatorTok{$}\NormalTok{node)}

\CommentTok{#remove space}
\NormalTok{unique}\OperatorTok{$}\NormalTok{node <-}\StringTok{ }\KeywordTok{gsub}\NormalTok{(}
  \StringTok{'}\CharTok{\textbackslash{}\textbackslash{}}\StringTok{s+'}\NormalTok{, }\StringTok{''}\NormalTok{, unique}\OperatorTok{$}\NormalTok{node}
\NormalTok{  )}
\end{Highlighting}
\end{Shaded}

\hypertarget{arrange-and-reshape-data}{%
\subsection{2. Arrange and reshape
data}\label{arrange-and-reshape-data}}

Note: each conduit has start and end coordinates and nodes, - so there
are duplicate rows for each conduit. We want to reshape this data to
have both xy and nodes in the same row.

\begin{Shaded}
\begin{Highlighting}[]
\CommentTok{#arrange by conduit name and descending elevation}
\NormalTok{arrange <-}\StringTok{ }\NormalTok{unique[}
  \KeywordTok{with}\NormalTok{(}
\NormalTok{  unique, }\KeywordTok{order}\NormalTok{(}
\NormalTok{    name, }
    \OperatorTok{-}\NormalTok{elevation, }
    \DataTypeTok{na.last=}\OtherTok{FALSE}\NormalTok{)}
\NormalTok{  ),}
\NormalTok{  ]}

\CommentTok{#reshape data}
\NormalTok{reshape_dt <-}\StringTok{ }\KeywordTok{dcast}\NormalTok{(}
  \KeywordTok{setDT}\NormalTok{(arrange), }
\NormalTok{  name }\OperatorTok{+}\StringTok{ }\NormalTok{roughness }\OperatorTok{~}\StringTok{ }\KeywordTok{rowid}\NormalTok{(name, }\DataTypeTok{prefix=}\StringTok{"node"}\NormalTok{), }
  \DataTypeTok{value.var=}\KeywordTok{c}\NormalTok{(}\StringTok{"node"}\NormalTok{, }\StringTok{"x"}\NormalTok{, }\StringTok{"y"}\NormalTok{, }\StringTok{"elevation"}\NormalTok{))}
\end{Highlighting}
\end{Shaded}

\hypertarget{length}{%
\subsection{3. Length}\label{length}}

We must now calculate the length of the conduits using the following
steps: a. Find distance ``length'' between xy coordinates of each
conduit using geosphere. b. Use difference in elevation to calculate
height c. Use pythag. theorem to calculate length

\begin{Shaded}
\begin{Highlighting}[]
\CommentTok{#part a -  distance (adjust code based on number of pairs. }
\CommentTok{#This dataset has 18 based on the longest conduit)}

\NormalTok{dist <-}\StringTok{ }\NormalTok{reshape_dt }\OperatorTok\StringTok{  }
\StringTok{  }\KeywordTok{rowwise}\NormalTok{(}
\NormalTok{  ) }\OperatorTok\StringTok{ }
\StringTok{  }\KeywordTok{mutate}\NormalTok{ (}
    \DataTypeTok{dist_m =} \KeywordTok{distm}\NormalTok{(}\KeywordTok{c}\NormalTok{(x_node1, y_node1), }
                   \KeywordTok{c}\NormalTok{(x_node2, y_node2), }
                   \DataTypeTok{fun =}\NormalTok{ distHaversine}
\NormalTok{                   )}
\NormalTok{    ) }\OperatorTok\StringTok{ }
\StringTok{  }\KeywordTok{mutate}\NormalTok{(}
    \DataTypeTok{dist_ft =}\NormalTok{ dist_m}\OperatorTok{*}\FloatTok{3.28084}
\NormalTok{    )}

\CommentTok{#part b and c - height then length}
\NormalTok{lengths <-}\StringTok{ }\NormalTok{dist }\OperatorTok\StringTok{  }
\StringTok{  }\KeywordTok{mutate}\NormalTok{ (}
    \DataTypeTok{length =} \KeywordTok{sqrt}\NormalTok{(}
\NormalTok{      (dist_ft)}\OperatorTok{^}\DecValTok{2} \OperatorTok{+}\StringTok{ }\NormalTok{(elevation_node1}\OperatorTok{-}\NormalTok{elevation_node2)}\OperatorTok{^}\DecValTok{2}\NormalTok{)}
\NormalTok{          )}\OperatorTok\StringTok{ }
\StringTok{  }\KeywordTok{rename}\NormalTok{ (}
    \DataTypeTok{from_node =}\NormalTok{ node_node1,}
    \DataTypeTok{to_node =}\NormalTok{ node_node2}
\NormalTok{  )}
\end{Highlighting}
\end{Shaded}

\hypertarget{file-output-for-conduits}{%
\subsection{4. File output for
conduits}\label{file-output-for-conduits}}

{[}CONDUITS{]}\\
;;Name From Node To Node Length Roughness InOffset OutOffset InitFlow
MaxFlow\\
;;---------- ---------- ---------- ---------- ---------- ----------
---------- ---------- ----------\\
** The channelized stream needs to connect with the stormwater
infrastructure via junctions/nodes in the urban region. It is feasible
to connect them manually by drawing in the stream conduit in SWMM.

Use roughness value 0.01, for concrete pipes found in Appendix A-8 pg.
184 of EPA manual

\begin{Shaded}
\begin{Highlighting}[]
\NormalTok{conduits <-}\StringTok{ }\NormalTok{lengths }\OperatorTok\StringTok{ }
\StringTok{  }\KeywordTok{mutate}\NormalTok{(}
    \DataTypeTok{roughness =} \KeywordTok{ifelse}\NormalTok{(}\KeywordTok{is.na}\NormalTok{(roughness), }\FloatTok{0.01}\NormalTok{, roughness)}
\NormalTok{    ) }\OperatorTok\StringTok{ }
\StringTok{  }\KeywordTok{select}\NormalTok{(}
\NormalTok{    name,}
\NormalTok{    from_node, }
\NormalTok{    to_node,}
\NormalTok{    length, }
\NormalTok{    roughness }
\NormalTok{    ) }

\NormalTok{conduits}\OperatorTok{$}\NormalTok{inoffset <-}\StringTok{ }\DecValTok{0}
\NormalTok{conduits}\OperatorTok{$}\NormalTok{outoffset <-}\StringTok{ }\DecValTok{0}
\NormalTok{conduits}\OperatorTok{$}\NormalTok{initflow <-}\StringTok{ }\DecValTok{0}
\NormalTok{conduits}\OperatorTok{$}\NormalTok{maxflow <-}\StringTok{ }\DecValTok{0}


\KeywordTok{write.csv}\NormalTok{(conduits,}\StringTok{"inp_conduits.csv"}\NormalTok{, }\DataTypeTok{row.names =} \OtherTok{FALSE}\NormalTok{)}
\end{Highlighting}
\end{Shaded}

\hypertarget{file-output-for-conduit-cross-sections}{%
\subsection{5. File output for conduit cross
sections}\label{file-output-for-conduit-cross-sections}}

{[}XSECTIONS{]}\\
;;Link Shape Geom1 Geom2 Geom3 Geom4 Barrels Culvert\\
;;-------------- ------------ ---------------- ----------

\begin{Shaded}
\begin{Highlighting}[]
\NormalTok{xsection_dt <-}\StringTok{ }\KeywordTok{merge}\NormalTok{(}
\NormalTok{  lengths, }
\NormalTok{  unique, }
  \DataTypeTok{by =} \StringTok{"name"}
\NormalTok{  ) }\OperatorTok\StringTok{ }
\StringTok{  }\KeywordTok{select}\NormalTok{ (}
\NormalTok{    name, }
\NormalTok{    length, }
\NormalTok{    shape, }
\NormalTok{    diameter, }
\NormalTok{    width, }
\NormalTok{    height}
\NormalTok{    )}
\end{Highlighting}
\end{Shaded}

a - concrete pipe dimensions

\begin{Shaded}
\begin{Highlighting}[]
\NormalTok{pipes<-}\StringTok{ }\NormalTok{xsection_dt }\OperatorTok\StringTok{ }
\StringTok{  }\KeywordTok{filter}\NormalTok{(}
\NormalTok{    shape }\OperatorTok{==}\StringTok{ "CIRCULAR"}
\NormalTok{    ) }\OperatorTok\StringTok{ }
\StringTok{  }\KeywordTok{rename}\NormalTok{(}
    \DataTypeTok{geom1 =}\NormalTok{ diameter,}
    \DataTypeTok{link =}\NormalTok{ name}
\NormalTok{    ) }\OperatorTok
\StringTok{  }\KeywordTok{mutate}\NormalTok{(}
    \DataTypeTok{geom1 =} \KeywordTok{ifelse}\NormalTok{(}\KeywordTok{is.na}\NormalTok{(geom1), }\StringTok{"23"}\NormalTok{, geom1)}
\NormalTok{    ) }\OperatorTok\StringTok{ }
\StringTok{  }\KeywordTok{mutate}\NormalTok{(}
    \DataTypeTok{geom1 =} \KeywordTok{ifelse}\NormalTok{((geom1}\OperatorTok{==}\StringTok{"Other"}\NormalTok{), }\StringTok{"23"}\NormalTok{, (geom1))}
\NormalTok{    )}\OperatorTok\StringTok{ }
\StringTok{  }\KeywordTok{select}\NormalTok{(}
\NormalTok{    link, }
\NormalTok{    shape,}
\NormalTok{    length,}
\NormalTok{    geom1}
\NormalTok{  )}
\NormalTok{pipes}\OperatorTok{$}\NormalTok{geom2 <-}\StringTok{ }\DecValTok{0}
\NormalTok{pipes}\OperatorTok{$}\NormalTok{geom3 <-}\StringTok{ }\DecValTok{0}
\NormalTok{pipes}\OperatorTok{$}\NormalTok{geom4 <-}\StringTok{ }\DecValTok{0}
\NormalTok{pipes}\OperatorTok{$}\NormalTok{barrels <-}\StringTok{ }
\NormalTok{pipes}\OperatorTok{$}\NormalTok{culvert <-}\StringTok{ }\DecValTok{0}
\end{Highlighting}
\end{Shaded}

b - box culverts, ditches, and ``other'' conduit dimensions

\begin{Shaded}
\begin{Highlighting}[]
\NormalTok{ditch_box <-}\StringTok{ }\NormalTok{xsection_dt }\OperatorTok\StringTok{ }
\StringTok{  }\KeywordTok{filter}\NormalTok{(}
\NormalTok{    shape }\OperatorTok{==}\StringTok{ "RECT_CLOSED"} \OperatorTok{|}\StringTok{ }\NormalTok{shape }\OperatorTok{==}\StringTok{ "RECT_OPEN"}
\NormalTok{    ) }\OperatorTok\StringTok{ }
\StringTok{  }\KeywordTok{rename}\NormalTok{ (}
    \DataTypeTok{geom1 =}\NormalTok{ width, }
    \DataTypeTok{geom2 =}\NormalTok{ height,}
    \DataTypeTok{link =}\NormalTok{ name}
\NormalTok{    ) }\OperatorTok\StringTok{ }
\StringTok{  }\KeywordTok{mutate}\NormalTok{(}
    \DataTypeTok{geom1 =} \KeywordTok{ifelse}\NormalTok{(}\KeywordTok{is.na}\NormalTok{(geom1), }\DecValTok{5}\NormalTok{, geom1)}
\NormalTok{    ) }\OperatorTok\StringTok{ }
\StringTok{  }\KeywordTok{mutate}\NormalTok{(}\DataTypeTok{geom2 =} \KeywordTok{ifelse}\NormalTok{(}\KeywordTok{is.na}\NormalTok{(geom2), }\DecValTok{5}\NormalTok{, geom2)}
\NormalTok{         ) }\OperatorTok\StringTok{ }
\StringTok{  }\KeywordTok{select}\NormalTok{( }
\NormalTok{    link, }
\NormalTok{    shape, }
\NormalTok{    length,}
\NormalTok{    geom1,}
\NormalTok{    geom2}
    
\NormalTok{  )}
\NormalTok{ditch_box}\OperatorTok{$}\NormalTok{geom3 <-}\StringTok{ }\DecValTok{0}
\NormalTok{ditch_box}\OperatorTok{$}\NormalTok{geom4 <-}\StringTok{ }\DecValTok{0}
\NormalTok{ditch_box}\OperatorTok{$}\NormalTok{barrels <-}\StringTok{ }
\NormalTok{ditch_box}\OperatorTok{$}\NormalTok{culvert <-}\StringTok{ }\DecValTok{0}
\end{Highlighting}
\end{Shaded}

c - channel (channelized stream) dimenssions

\begin{Shaded}
\begin{Highlighting}[]
\NormalTok{channel <-}\StringTok{ }\NormalTok{xsection_dt }\OperatorTok\StringTok{ }
\StringTok{  }\KeywordTok{filter}\NormalTok{(}
    \KeywordTok{is.na}\NormalTok{(shape)}
\NormalTok{    ) }\OperatorTok\StringTok{ }
\StringTok{  }\KeywordTok{mutate}\NormalTok{(}
    \DataTypeTok{shape =} \KeywordTok{ifelse}\NormalTok{(}\KeywordTok{is.na}\NormalTok{(shape), }\StringTok{"TRAPEZOIDAL"}\NormalTok{, shape)}
\NormalTok{    ) }\OperatorTok\StringTok{ }
\StringTok{  }\KeywordTok{rename}\NormalTok{(}
    \DataTypeTok{geom1 =}\NormalTok{ width, }
    \DataTypeTok{geom2 =}\NormalTok{ height,}
    \DataTypeTok{link =}\NormalTok{ name}
\NormalTok{    ) }\OperatorTok\StringTok{ }
\StringTok{  }\KeywordTok{mutate}\NormalTok{(}
    \DataTypeTok{geom1 =} \KeywordTok{ifelse}\NormalTok{(}\KeywordTok{is.na}\NormalTok{(geom1), }\DecValTok{30}\NormalTok{, geom1)}
\NormalTok{    ) }\OperatorTok\StringTok{ }
\StringTok{  }\KeywordTok{mutate}\NormalTok{(}
    \DataTypeTok{geom2 =} \KeywordTok{ifelse}\NormalTok{(}\KeywordTok{is.na}\NormalTok{(geom2), }\DecValTok{10}\NormalTok{, geom2)}
\NormalTok{    ) }\OperatorTok\StringTok{ }
\StringTok{  }\KeywordTok{select}\NormalTok{(}
\NormalTok{     link, }
\NormalTok{    shape,}
\NormalTok{    length,}
\NormalTok{    geom1,}
\NormalTok{    geom2}
\NormalTok{    )}
\CommentTok{#Geom3 and Geom4 are side lopes, which literature indicate vary from 1/1 to 1/2.  }
\CommentTok{#Ive seen side slopes perpendicular to the ground, especially where homes are built.}
\NormalTok{channel}\OperatorTok{$}\NormalTok{geom3 <-}\StringTok{ }\DecValTok{1}
\NormalTok{channel}\OperatorTok{$}\NormalTok{geom4 <-}\StringTok{ }\DecValTok{1}
\NormalTok{channel}\OperatorTok{$}\NormalTok{barrels <-}\StringTok{ }
\NormalTok{channel}\OperatorTok{$}\NormalTok{culvert <-}\StringTok{ }\DecValTok{0}
\end{Highlighting}
\end{Shaded}

e- bind all tables

\begin{Shaded}
\begin{Highlighting}[]
\NormalTok{xsections_df <-}\StringTok{ }\KeywordTok{rbind}\NormalTok{(}
\NormalTok{  pipes, }
\NormalTok{  ditch_box, }
\NormalTok{  channel}
\NormalTok{  ) }\OperatorTok\StringTok{ }
\StringTok{  }\KeywordTok{select}\NormalTok{(}
\NormalTok{    link, }
\NormalTok{    shape, }
\NormalTok{    geom1, }
\NormalTok{    geom2, }
\NormalTok{    geom3, }
\NormalTok{    geom4, }
\NormalTok{    barrels, }
\NormalTok{    culvert, }
\NormalTok{    length}
\NormalTok{    ) }\OperatorTok\StringTok{  }
\StringTok{  }\KeywordTok{distinct}\NormalTok{(}
\NormalTok{    link, }
\NormalTok{    shape, }
\NormalTok{    geom1, }
\NormalTok{    geom2, }
    \DataTypeTok{.keep_all =} \OtherTok{TRUE}
\NormalTok{    )}

\NormalTok{xsections <-}\StringTok{ }\KeywordTok{rbind}\NormalTok{(}
\NormalTok{  pipes, }
\NormalTok{  ditch_box, }
\NormalTok{  channel}
\NormalTok{  ) }\OperatorTok\StringTok{ }
\StringTok{  }\KeywordTok{select}\NormalTok{(}
\NormalTok{    link, }
\NormalTok{    shape, }
\NormalTok{    geom1, }
\NormalTok{    geom2, }
\NormalTok{    geom3, }
\NormalTok{    geom4, }
\NormalTok{    barrels, }
\NormalTok{    culvert}
\NormalTok{    ) }\OperatorTok\StringTok{ }
\StringTok{  }\KeywordTok{distinct}\NormalTok{(}
\NormalTok{    link, }
\NormalTok{    shape, }
\NormalTok{    geom1, }
\NormalTok{    geom2, }
    \DataTypeTok{.keep_all =} \OtherTok{TRUE}
\NormalTok{    )}

\KeywordTok{write.csv}\NormalTok{(xsections,}\StringTok{"inp_xsections.csv"}\NormalTok{, }\DataTypeTok{row.names =} \OtherTok{FALSE}\NormalTok{)}
\end{Highlighting}
\end{Shaded}

\hypertarget{junctions-and-coordinatesvertices}{%
\subsection{6. Junctions and
coordinates/vertices}\label{junctions-and-coordinatesvertices}}

{[}JUNCTIONS{]}\\
;;Name Elevation MaxDepth InitDepth SurDepth Aponded\\
;;-------------- ---------- ---------- ----------

\begin{Shaded}
\begin{Highlighting}[]
\NormalTok{junctions <-}\StringTok{ }\NormalTok{unique }\OperatorTok\StringTok{ }\KeywordTok{filter}\NormalTok{(}
\NormalTok{    structure }\OperatorTok{!=}\StringTok{ "Inlet/Outlet"} \OperatorTok{|}\StringTok{ }\KeywordTok{is.na}\NormalTok{(structure)}
\NormalTok{    ) }\OperatorTok\StringTok{ }
\StringTok{  }\KeywordTok{select}\NormalTok{(}
\NormalTok{    node, }
\NormalTok{    elevation}
\NormalTok{    ) }\OperatorTok\StringTok{ }
\StringTok{  }\KeywordTok{rename}\NormalTok{(}
    \DataTypeTok{name =}\NormalTok{ node}
\NormalTok{    ) }\OperatorTok\StringTok{ }
\StringTok{  }\KeywordTok{distinct}\NormalTok{(}
\NormalTok{    name,}
    \DataTypeTok{.keep_all =} \OtherTok{TRUE}
\NormalTok{  )}
\NormalTok{junctions}\OperatorTok{$}\NormalTok{maxdepth <-}\StringTok{ }\DecValTok{0}
\NormalTok{junctions}\OperatorTok{$}\NormalTok{initdepth <-}\StringTok{ }\DecValTok{0}
\NormalTok{junctions}\OperatorTok{$}\NormalTok{surdepth <-}\StringTok{ }\DecValTok{0}
\NormalTok{junctions}\OperatorTok{$}\NormalTok{aponded <-}\StringTok{ }\DecValTok{0}

\CommentTok{#filter for structure type in here so we can designate which junctions are outfalls}
\KeywordTok{write.csv}\NormalTok{(junctions,}\StringTok{"inp_junctions.csv"}\NormalTok{, }\DataTypeTok{row.names =} \OtherTok{FALSE}\NormalTok{)}
\end{Highlighting}
\end{Shaded}

Here is a csv that will help determine what junction to route each
subcatchment's outlet to.

\begin{Shaded}
\begin{Highlighting}[]
\NormalTok{route_to <-}\StringTok{ }\NormalTok{unique }\OperatorTok\StringTok{ }
\StringTok{  }\KeywordTok{select}\NormalTok{(}
\NormalTok{    node, }
\NormalTok{    elevation,}
\NormalTok{    subc}
\NormalTok{    )}

\NormalTok{route_to[}
  \KeywordTok{with}\NormalTok{(}
\NormalTok{  route_to, }\KeywordTok{order}\NormalTok{(}
\NormalTok{    subc, }
    \OperatorTok{-}\NormalTok{elevation, }
    \DataTypeTok{na.last=}\OtherTok{FALSE}\NormalTok{)}
\NormalTok{  ),}
\NormalTok{  ]}

\NormalTok{route_to2 <-unique }\OperatorTok\StringTok{ }
\StringTok{  }\KeywordTok{group_by}\NormalTok{(}
\NormalTok{    subc}
\NormalTok{    ) }\OperatorTok\StringTok{ }
\StringTok{  }\KeywordTok{mutate}\NormalTok{(}
    \DataTypeTok{rank =} \KeywordTok{row_number}\NormalTok{(subc)}
\NormalTok{    ) }\OperatorTok\StringTok{ }
\StringTok{  }\KeywordTok{filter}\NormalTok{(}
\NormalTok{    rank }\OperatorTok{==}\StringTok{ }\DecValTok{1} \OperatorTok{|}\StringTok{ }\KeywordTok{is.na}\NormalTok{(rank)}
\NormalTok{    ) }\OperatorTok\StringTok{ }
\StringTok{  }\KeywordTok{select}\NormalTok{(}
\NormalTok{    subc,}
\NormalTok{    node}
\NormalTok{    )}

\KeywordTok{write.csv}\NormalTok{(route_to2,}\StringTok{"subc_outlet.csv"}\NormalTok{, }\DataTypeTok{row.names =} \OtherTok{FALSE}\NormalTok{)}
\end{Highlighting}
\end{Shaded}

\hypertarget{outfalls}{%
\subsection{7. Outfalls}\label{outfalls}}

{[}OUTFALLS{]}\\
;;Name Elevation Type Stage Data Gated Route To\\
;;-------------- ---------- ---------- ---------------- --------
----------------

\begin{Shaded}
\begin{Highlighting}[]
\NormalTok{outfalls <-}\StringTok{ }\NormalTok{unique }\OperatorTok\StringTok{ }
\StringTok{   }\KeywordTok{filter}\NormalTok{(}
\NormalTok{    structure }\OperatorTok{==}\StringTok{ "Inlet/Outlet"}
\NormalTok{    ) }\OperatorTok\StringTok{ }
\StringTok{  }\KeywordTok{select}\NormalTok{(}
\NormalTok{    node, }
\NormalTok{    elevation}
\NormalTok{    ) }\OperatorTok\StringTok{ }
\StringTok{  }\KeywordTok{rename}\NormalTok{(}
    \DataTypeTok{name =}\NormalTok{ node}
\NormalTok{    ) }\OperatorTok\StringTok{ }
\StringTok{  }\KeywordTok{distinct}\NormalTok{(}
\NormalTok{    name,}
    \DataTypeTok{.keep_all =} \OtherTok{TRUE}
\NormalTok{  )}
\NormalTok{outfalls}\OperatorTok{$}\NormalTok{type <-}\StringTok{ "FREE"}
\NormalTok{outfalls}\OperatorTok{$}\NormalTok{stagedata <-}\StringTok{ }\OtherTok{NA}
\NormalTok{outfalls}\OperatorTok{$}\NormalTok{gated <-}\StringTok{ "NO"}
\NormalTok{outfalls}\OperatorTok{$}\NormalTok{routeto <-}\StringTok{ }\OtherTok{NA}

\CommentTok{#you will need to check these structures in SWMM. }
\CommentTok{#If these structures are at an end of a conduit, then they are outfalls, }
\CommentTok{#otherwise you can quickly turn them back into a junction in SWMM.}

\KeywordTok{write.csv}\NormalTok{(outfalls,}\StringTok{"inp_outfalls.csv"}\NormalTok{, }\DataTypeTok{row.names =} \OtherTok{FALSE}\NormalTok{)}
\end{Highlighting}
\end{Shaded}

\hypertarget{coordinates}{%
\subsection{8. Coordinates}\label{coordinates}}

{[}COORDINATES{]}\\
;;Node X-Coord Y-Coord\\
;;-------------- ------------------ ------------------

SWMM takes x and y that are in decimal degrees (lat and long).

\begin{Shaded}
\begin{Highlighting}[]
\NormalTok{coordinates <-}\StringTok{ }\NormalTok{unique }\OperatorTok\StringTok{ }
\StringTok{  }\KeywordTok{select}\NormalTok{(}
\NormalTok{    node, }
\NormalTok{    x, }
\NormalTok{    y}
\NormalTok{    ) }\OperatorTok\StringTok{ }
\StringTok{  }\KeywordTok{distinct}\NormalTok{(}
\NormalTok{    node,}
\NormalTok{    x,}
\NormalTok{    y,}
    \DataTypeTok{.keep_all =} \OtherTok{TRUE}
\NormalTok{  )}

\KeywordTok{write.csv}\NormalTok{(coordinates,}\StringTok{"inp_coordinates.csv"}\NormalTok{, }\DataTypeTok{row.names =} \OtherTok{FALSE}\NormalTok{)}
\end{Highlighting}
\end{Shaded}

\hypertarget{vertices}{%
\subsection{9. Vertices}\label{vertices}}

{[}VERTICES{]}\\
;;Link X-Coord Y-Coord\\
;;-------------- ------------------ ------------------

\begin{Shaded}
\begin{Highlighting}[]
\NormalTok{vertices <-}\StringTok{ }\NormalTok{vertices_dt }\OperatorTok
\StringTok{  }\KeywordTok{select}\NormalTok{(}
\NormalTok{    objectid,}
\NormalTok{    point_x,}
\NormalTok{    point_y}
\NormalTok{  ) }\OperatorTok\StringTok{ }
\StringTok{  }\KeywordTok{rename}\NormalTok{(}
   \DataTypeTok{link =}\NormalTok{ objectid,}
   \DataTypeTok{x =}\NormalTok{ point_x,}
   \DataTypeTok{y =}\NormalTok{ point_y}
\NormalTok{    ) }\OperatorTok\StringTok{ }
\StringTok{  }\KeywordTok{distinct}\NormalTok{(}
\NormalTok{    link,}
\NormalTok{    x,}
\NormalTok{    y,}
    \DataTypeTok{.keep_all =} \OtherTok{TRUE}
\NormalTok{  )}
\NormalTok{vertices}\OperatorTok{$}\NormalTok{c <-}\StringTok{ "C"}

\CommentTok{#conduits}
\NormalTok{vertices}\OperatorTok{$}\NormalTok{link <-}\StringTok{ }\KeywordTok{apply}\NormalTok{(}
\NormalTok{  vertices,}\DecValTok{1}\NormalTok{,}\ControlFlowTok{function}\NormalTok{(x) }\KeywordTok{paste}\NormalTok{((x[}\DecValTok{4}\NormalTok{]), }\DataTypeTok{sep =}\StringTok{""}\NormalTok{,(x[}\DecValTok{1}\NormalTok{]))}
\NormalTok{  ) }

\CommentTok{#remove space}
\NormalTok{vertices}\OperatorTok{$}\NormalTok{link <-}\StringTok{ }\KeywordTok{gsub}\NormalTok{(}
  \StringTok{'}\CharTok{\textbackslash{}\textbackslash{}}\StringTok{s+'}\NormalTok{, }\StringTok{''}\NormalTok{, vertices}\OperatorTok{$}\NormalTok{link}
\NormalTok{  ) }
 
\NormalTok{  vertices1 <-}\StringTok{ }\NormalTok{vertices }\OperatorTok\StringTok{ }
\StringTok{  }\KeywordTok{select}\NormalTok{(}
\NormalTok{    link,}
\NormalTok{    x,}
\NormalTok{    y}
\NormalTok{  )}

\KeywordTok{write.csv}\NormalTok{(vertices1,}\StringTok{"inp_vertices.csv"}\NormalTok{, }\DataTypeTok{row.names =} \OtherTok{FALSE}\NormalTok{)}
\end{Highlighting}
\end{Shaded}


\end{document}
